\documentclass{resume}
\author{Paweł Żak}

\begin{document}
 \section{Sectioning}

  The document class allows to divide your CV into logical parts using
  standard \LaTeX sectioning commands. Inside sections one can create
  normal paragraphs and use other environments, e.g. \emph{bfseries}:
 
  \begin{bfseries}
   Here's where you can write a brief summary description of your person written
   using \emph{bfseries}
  \end{bfseries}

  Note, that this document class uses \emph{parskip} package, so you can use
  blank lines to separate paragraphs. However, the paragraph skip has been
  reduced to 2pt, so the distance between the paragraphs is barely noticeable
  and allows to fit more information onto a single page.
 
 \section{``factstable''}

  A \emph{factstable} environment allows to produce two-column output with label
  of a personal "fact" on the left and value on the right. It is a wrapper
  around the \emph{longtable} environment, so it's possible to use extensions
  from that package. For more informations see
  \href{http://mirrors.ibiblio.org/CTAN/macros/latex/required/tools/longtable.pdf}{longtable.pdf}

  Below you can see an example of a section featuring simple \emph{factstable}
  with a handful of personal details.

 \section{Personal Details}

  \begin{factstable}
   Name    & John Doe \\
   Address & 524 North Avenue, 68419 Panama, NE \\
   E-mail  & 9glg9om8p4b@temporary-mail.net
  \end{factstable}

 \section{``fact''}

  The \emph{fact} environment builds on top of \emph{factstable}, allowing to
  describe more complex events in one's professional life, e.g. projects.

 \section{Work Experience}

  \begin{factstable}
   Since 01.2020 \newline
   Baden, Switzerland
   &
    \begin{fact}[LaTeX Developer]
     In 2020 I decided to publish my custom document class for authoring résumés
     on GitHub. It had turned out to be an immediate success.
    \end{fact}
   \\
  \end{factstable}

\end{document}
